% ==========================================
%              MAIN CONTENT
% ==========================================

\section{Word-Like Formatting Features}

\subsection{Text Styling}
LaTeX makes basic text formatting simple:
\begin{itemize}
    \item \textbf{This text is Bold} 
    \item \textit{This text is Italic} 
    \item \textbf{\textit{This text is Bold and Italic}} 
    \item \underline{This text is Underlined} 
\end{itemize}

\subsection{Text Sizes}
\begin{itemize}
    \item {\tiny This is tiny text} 
    \item {\scriptsize This is scriptsize text} 
    \item {\footnotesize This is footnotesize text} 
    \item {\small This is small text} 
    \item {\normalsize This is normalsize text (Default)} 
    \item {\large This is large text} 
    \item {\Large This is Large text} 
    \item {\LARGE This is LARGE text} 
    \item {\huge This is huge text} 
    \item {\Huge This is Huge text} 
\end{itemize}

\clearpage

\section{Tables}

\subsection{Standard Professional Table}
\begin{table}[ht]
    \centering
    \caption{A standard academic table.}
    \label{tab:standard}
    \begin{tabular}{llr} 
        \toprule
        \textbf{Item Name} & \textbf{Category} & \textbf{Price (\$)} \\ 
        \midrule
        Apple              & Fruit             & 1.50 \\
        Sourdough Bread    & Bakery            & 4.00 \\
        Milk (1 Gallon)    & Dairy             & 3.20 \\
        \bottomrule
    \end{tabular}
\end{table}

\subsection{Alternating Row Color Table}
\begin{table}[ht]
    \centering
    \caption{Table with alternating row colors (Zebra striping).}
    \label{tab:colored}
    \rowcolors{2}{white}{codebg} 
    \begin{tabular}{p{4cm} c c} 
        \toprule
        \textbf{Employee} & \textbf{ID} & \textbf{Department} \\ 
        \midrule
        John Doe          & 1001        & Engineering \\
        Jane Smith        & 1002        & Marketing \\
        Bob Johnson       & 1003        & Design \\
        Alice Williams    & 1004        & Engineering \\
        \bottomrule
    \end{tabular}
\end{table}

\clearpage

\section{Images and Layouts}

\subsection{Standard Image}
\begin{figure}[H]
  \centering
  \includegraphics[width=0.6\textwidth]{example-image-a} 
  \captionsetup{style=mycustom}
  \caption{Standard single image.}
  \label{fig:single}
  \vspace{1mm} 
  {\footnotesize\textcolor{gray}{\textit{Source: Generated by LaTeX}}}
\end{figure}

\subsection{Stacked Images (Subfigures)}
\begin{figure}[H]
    \centering
    \begin{subfigure}[b]{0.45\textwidth}
        \centering
        \includegraphics[width=\textwidth]{example-image-b}
        \caption{Top or Left Image}
        \label{fig:stack1}
    \end{subfigure}
    \hfill 
    \begin{subfigure}[b]{0.45\textwidth}
        \centering
        \includegraphics[width=\textwidth]{example-image-c}
        \caption{Bottom or Right Image}
        \label{fig:stack2}
    \end{subfigure}
    \caption{Two images stacked side-by-side.}
    \label{fig:stacked}
\end{figure}

\subsection{Custom Styled Figure}
\begin{figure}[H]
  \begin{center}
      \rule{\textwidth}{0.8pt}  
  \end{center}
  \captionsetup{style=mycustom} 
  \caption{Extended detail overview of the figure shown below...}
  \label{fig:custom_example}
  \begin{center}
      % Uncomment to use your file:
      % \includegraphics[width=0.8\textwidth]{resources/2_bLine_regressplot.png}
      \includegraphics[width=0.8\textwidth]{example-image}
  \end{center}
  \vspace{1mm} 
  {\footnotesize\textcolor{gray}{\textit{Source: --Source of the Graph--}}}
\end{figure}