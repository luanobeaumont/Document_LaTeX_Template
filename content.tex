% ==========================================
%              MAIN CONTENT
% ==========================================

\section{Word-Like Formatting Features}

\subsection{Text Styling}
LaTeX makes basic text formatting simple:
\begin{itemize}
    \item \textbf{This text is Bold} 
    \item \textit{This text is Italic} 
    \item \textbf{\textit{This text is Bold and Italic}} 
    \item \underline{This text is Underlined} 
\end{itemize}

\subsection{Text Sizes}
\begin{itemize}
    \item {\tiny This is tiny text} 
    \item {\scriptsize This is scriptsize text} 
    \item {\footnotesize This is footnotesize text} 
    \item {\small This is small text} 
    \item {\normalsize This is normalsize text (Default)} 
    \item {\large This is large text} 
    \item {\Large This is Large text} 
    \item {\LARGE This is LARGE text} 
    \item {\huge This is huge text} 
    \item {\Huge This is Huge text} 
\end{itemize}

\clearpage

\section{Classified Data Logs}

\subsection{Example: C Encryption Routine}
% 1. The Source Code
\begin{ciacode}{C}{CONFIDENTIAL: ENCRYPTION PROTOCOL}
#include <stdio.h>
#include <string.h>

// TOP SECRET ROUTINE
int main() {
    int target_id = 9921;
    char *status = "CLASSIFIED";
    
    // Check clearance level
    if (target_id > 9000) {
        printf("[+] Priority Target Identified: %s\n", status);
        printf("[*] Initializing assets...\n");
    } else {
        printf("(!) Access Denied.\n");
    }
    return 0;
}
\end{ciacode}

% 2. The Output Log (using new cblkout)
\begin{cblkout}{TERMINAL OUTPUT: RUN_1}
user@agency-term:~$ gcc encrypt.c -o encrypt
user@agency-term:~$ ./encrypt
[+] Priority Target Identified: CLASSIFIED
[*] Initializing assets...
user@agency-term:~$ _
\end{cblkout}

\subsection{Example: R Statistical Analysis}
% 1. The Source Code
\begin{ciacode}{R}{ANALYSIS: SUSPECT MOVEMENT DATA}
# Load the surveillance dataset
data <- read.csv("suspects.csv")

# Filter for high-risk targets (> 90%)
targets <- subset(data, risk_level > 0.9)

# Display summary stats
print(summary(targets$risk_level))
\end{ciacode}

% 2. The Output Log
\begin{cblkout}{R CONSOLE EXPORT}
   Min. 1st Qu.  Median    Mean 3rd Qu.    Max. 
  0.910   0.925   0.950   0.948   0.980   0.999 
\end{cblkout}

\vspace{1em}
This is an example of \inlinecode{inline code} using the same theme.

\clearpage

\section{Tables}

\subsection{Standard Professional Table}
\begin{table}[ht]
    \centering
    \caption{A standard academic table.}
    \label{tab:standard}
    \begin{tabular}{llr} 
        \toprule
        \textbf{Item Name} & \textbf{Category} & \textbf{Price (\$)} \\ 
        \midrule
        Apple              & Fruit             & 1.50 \\
        Sourdough Bread    & Bakery            & 4.00 \\
        Milk (1 Gallon)    & Dairy             & 3.20 \\
        \bottomrule
    \end{tabular}
\end{table}

\subsection{Alternating Row Color Table}
\begin{table}[ht]
    \centering
    \caption{Table with alternating row colors (Zebra striping).}
    \label{tab:colored}
    
    \begin{NiceTabular}{p{4cm} c c}
        \CodeBefore
            \rowcolors{2}{codebg}{white}
        \Body
            \toprule
            \textbf{Employee} & \textbf{ID} & \textbf{Department} \\ 
            \midrule
            John Doe          & 1001        & Engineering \\
            Jane Smith        & 1002        & Marketing \\
            Bob Johnson       & 1003        & Design \\
            Alice Williams    & 1004        & Engineering \\
            \bottomrule
    \end{NiceTabular}
\end{table}

\clearpage

\section{Images and Layouts}

\subsection{Standard Image}
\begin{figure}[H]
  \centering
  \includegraphics[width=0.6\textwidth]{example-image-a} 
  \captionsetup{style=mycustom}
  \caption{Standard single image.}
  \label{fig:single}
  \vspace{1mm} 
  {\footnotesize\textcolor{docgray}{\textit{Source: Generated by LaTeX}}}
\end{figure}

\subsection{Stacked Images (Subfigures)}
\begin{figure}[H]
    \centering
    \begin{subfigure}[b]{0.45\textwidth}
        \centering
        \includegraphics[width=\textwidth]{example-image-b}
        \caption{Top or Left Image}
        \label{fig:stack1}
    \end{subfigure}
    \hfill 
    \begin{subfigure}[b]{0.45\textwidth}
        \centering
        \includegraphics[width=\textwidth]{example-image-c}
        \caption{Bottom or Right Image}
        \label{fig:stack2}
    \end{subfigure}
    \caption{Two images stacked side-by-side.}
    \label{fig:stacked}
\end{figure}

\subsection{Custom Styled Figure}
\begin{figure}[H]
  \begin{center}
      \rule{\textwidth}{0.8pt}  
  \end{center}
  \captionsetup{style=mycustom} 
  \caption{Extended detail overview of the figure shown below...}
  \label{fig:custom_example}
  \begin{center}
      \includegraphics[width=0.8\textwidth]{example-image}
  \end{center}
  \vspace{1mm} 
  {\footnotesize\textcolor{docgray}{\textit{Source: --Source of the Graph--}}}
\end{figure}